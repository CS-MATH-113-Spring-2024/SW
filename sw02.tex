\documentclass{exam}
\title{CS113/DISCRETE MATHEMATICS-SPRING 2024}
\author{Worksheet 2}
\date{Topic: Logic And Proofs}
\begin{document}
\maketitle

\begin{center}
\fbox{\fbox{\parbox{5.5in}{\centering Use the given tables of laws/Truth tables to construct equivalences between  compound propositions.
Happy Learning!}}}
\end{center}

\vspace{5mm}
\makebox[0.75\textwidth]{Student's Name and ID:\enspace\hrulefill}

\vspace{5mm}
\makebox[0.75\textwidth]{Instructor’s name:\enspace\hrulefill}

\section{Laws Of Logical Equivalences: }
\vspace{5mm}
\begin{center}
\begin{table}[h]
\centering
\begin{tabular}{|c|c|}
\hline
Equivalence & Name \\
\hline
$p \land T \equiv p$ & Identity laws \\
$p \lor F \equiv p$ & \\
\hline
$p \lor T \equiv T$ & Domination laws \\
$p \land F \equiv F$ & \\
\hline
$p \lor p \equiv p$ & Idempotent laws \\
$p \land p \equiv p$ & \\
\hline
$\neg (\neg p) \equiv p$ & Double negation law \\
\hline
$p \lor q \equiv q \lor p$ & Commutative laws \\
$p \land q \equiv q \land p$ & \\
\hline
$(p \lor q) \lor r \equiv p \lor (q \lor r)$ & Associative laws \\
$(p \land q) \land r \equiv p \land (q \land r)$ & \\
\hline
$p \lor (q \land r) \equiv (p \lor q) \land (p \lor r)$ & Distributive laws \\
$p \land (q \lor r) \equiv (p \land q) \lor (p \land r)$ & \\
\hline
$\neg (p \land q) \equiv \neg p \lor \neg q$ & De Morgan’s laws \\
$\neg (p \lor q) \equiv \neg p \land \neg q$ & \\
\hline
$p \lor (p \land q) \equiv p$ & Absorption laws \\
$p \land (p \lor q) \equiv p$ & \\
\hline
$p \lor \neg p \equiv T$ & Negation laws \\
$p \land \neg p \equiv F$ & \\
\hline
\end{tabular}
\end{table}

\end{center}
\vspace{5mm}
\begin{center}
\begin{tabular}{|c|c|}
  \hline
  Equivalence & Rule \\
  \hline
  Conditional Law 1 & $p \rightarrow q \equiv \lnot p \lor q$ \\
  Conditional Law 2 & $p \rightarrow q \equiv \lnot q \rightarrow \lnot p$ \\
  Conditional Law 3 & $p \lor q \equiv \lnot p \rightarrow q$ \\
  Conditional Law 4 & $p \land q \equiv \lnot (p \rightarrow \lnot q)$ \\
  Conditional Law 5 & $\lnot (p \rightarrow q) \equiv p \land \lnot q$ \\
  Conditional Law 6 & $(p \rightarrow q) \land (p \rightarrow r) \equiv p \rightarrow (q \land r)$ \\
  Conditional Law 7 & $(p \rightarrow r) \land (q \rightarrow r) \equiv (p \lor q) \rightarrow r$ \\
  Conditional Law 8 & $(p \rightarrow q) \lor (p \rightarrow r) \equiv p \rightarrow (q \lor r)$ \\
  Conditional Law 9 & $(p \rightarrow r) \lor (q \rightarrow r) \equiv (p \land q) \rightarrow r$ \\
  \hline
\end{tabular}
\end{center}
\vspace{5mm}
\begin{center}
    \begin{tabular}{|c|c|}
  \hline
  Equivalence & Rule \\
  \hline
  Biconditional Law 1 & $p \leftrightarrow q \equiv (p \rightarrow q) \land (q \rightarrow p)$ \\
  Biconditional Law 2 & $p \leftrightarrow q \equiv \lnot p \leftrightarrow \lnot q$ \\
  Biconditional Law 3 & $p \leftrightarrow q \equiv (p \land q) \lor (\lnot p \land \lnot q)$ \\
  Biconditional Law 4 & $\lnot (p \leftrightarrow q) \equiv p \leftrightarrow \lnot q$ \\
  \hline
\end{tabular}

\end{center}


\begin{questions}
\question Show that following conditional statements are tautologies. (use laws of equivalences and not the truth table to prove it.)
\begin{parts}
\part
\[(\neg p \land (p \lor q)) \rightarrow q\]
\vspace{4in}

\part
\[((p \rightarrow q) \land (q \rightarrow r)) \rightarrow (p \rightarrow r)\]
\vspace{4in}

\part
\[(p \land (p \rightarrow q)) \rightarrow q\]
\vspace{4in}

\part
\[((p \lor q) \land (p \rightarrow r) \land (q \rightarrow r)) \rightarrow r\]
\vspace{4in}
\end{parts}


\question Show using truth table that \((p \rightarrow r) \land (q \rightarrow r)\) and \((p \lor q) \rightarrow r\) are logically equivalent.


\vspace{9in}

\question Show using truth table that \((p \lor q) \land (\neg p \lor r) \rightarrow (q \lor r)\) is tautology. 

\vspace{9in}
\end{questions}
\end{document}
