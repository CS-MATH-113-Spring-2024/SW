\documentclass{exam}
\usepackage{amssymb}
\usepackage{amsmath}
\title{CS113/DISCRETE MATHEMATICS-SPRING 2024}
\author{Worksheet 10}
\date{Topic: Functions (Injective, Surjective, And Bijective Functions)}
\begin{document}
\maketitle

\begin{center}
\fbox{\fbox{\parbox{5.5in}{\centering Through this lesson, we will explore three important properties of functions: surjectivity (onto), injectivity (one-to-one), and bijectivity(one to one correspondence).  Happy Learning!}}}
\end{center}

\vspace{5mm}
\makebox[0.75\textwidth]{Student's Name and ID:\enspace\hrulefill}

\vspace{5mm}
\makebox[0.75\textwidth]{Instructor’s name:\enspace\hrulefill}


\vspace{5mm}


\begin{questions}


\question Prove that the function F : $Z \rightarrow Z$ defined as F(n) = n + 6 is a bijection.
\vspace{4in}



\question For each of the following functions, prove that the function is 1-1 or find an appropriate pair of points to show that the function is not 1-1:

\begin{parts}
\part
F: $R \rightarrow R$
\[
f(n) = 
\begin{cases}
    n^2, & \text{for } n \geq 0 \\
    -n^2, & \text{for } 0 \geq n 
\end{cases}
\]
\vspace{4in}

\part
F: $R \rightarrow R$
\[
f(x) = 
\begin{cases}
    x+1, & \text{for } x \in Q \\
    2x, & \text{for } x \notin Q 
\end{cases}
\]
\vspace{6in}

\part
F: $R \rightarrow R$
\[
f(x) = 
\begin{cases}
    3x+2, & \text{for } x \in Q \\
    x^3, & \text{for } x \notin Q 
\end{cases}
\]
\vspace{4in}

\part
F: $R \rightarrow R$
\[
f(n) = 
\begin{cases}
    n+1, & \text{for n odd } \\
    n^3, & \text{for n even } 
\end{cases}
\]
\vspace{6in}

\end{parts}


\question Show that no $F : \rightarrow R$ is both increasing and strictly decreasing.

\vspace{9in}


\end{questions}
\end{document}